
\documentclass[dvipdfmx, 12pt]{jarticle}


\usepackage[margin=20truemm]{geometry}
\usepackage{pdfpages}
\usepackage{graphicx}
\usepackage[absolute,overlay]{textpos} % 絶対配置
 




\begin{document}
\noindent
{\large 
\underline{公募No. R2026-65}  \\
}

\noindent
京都大学大学院理学研究科 小池春人
 

\section{研究提案}

「宇宙放射線環境が宇宙機に与える影響を評価するための計測手法やモデルの研究開発」
というテーマに対し、本提案では “非線形変動も再現可能な放射線帯高エネルギー電子フラックス
予測モデルの開発” を行いたいと考える。
現在、貴研究所で提供されている高エネルギー電子フラックスの予測
(Sakaguchi et al., (2013)に基づくモデル)は、
自己回帰(Auto Regression: AR)モデルとカルマンフィルタを組み合わせ、
最新の観測値を逐次的にデータ同化しながら状態を更新している。
このモデルではフラックスの1日値をインプットとして用いており、
数日スケールの線形的なトレンド再現に優れている。
一方で、1日より短時間のスケールで発生する非線形的な変動の再現は難しいという課題がある。
そのような短時間の変動の原因としては、太陽風不連続構造の到来、
コーラス波動による加速、ULF波動による動径方向拡散などが挙げられる。


\vspace{.5em}
本提案では、Sakaguchi et al. (2013) の枠組みを拡張し、数時間スケールの非線形変動までを
再現可能な予測モデルの構築を目指す。
具体的には、ARモデルが説明しきれない変動成分を
非線形回帰(Nonlinear Autoregressive Exogenous: NARX)モデルにより記述し、
拡張カルマンフィルタによって観測データを逐次データ同化し、
モデル状態を更新することで、非線形性を動的に反映させた予測を実現する。
放射線帯電子フラックスの非線形モデルとしては、Landis et al. (2021)などの例があるものの、
逐次データ同化の仕組みを備えていない点で、本提案のアプローチとは異なる。
さらに、本提案ではインプットパラメータの再検討を行う。
従来の予測モデルでは太陽風データや地磁気指数が利用されてきたが、
磁気圏内部の波動(電子スケール波動、ULF帯波動強度など)も短時間スケールのフラックス変動に
寄与すると考えられる。そこで、これらのパラメータを追加した場合に予測精度がどのように向上
するかを評価する。また、インプットの時間分解能の違いが予測性能に与える影響についても
調査する。

\vspace{.5em}
以上の取り組みにより、1日以下の非線形的な短時間変動を再現可能な放射線帯高エネルギー
電子フラックス予測モデルを構築し、従来モデルの拡張を目指す。
宇宙機への影響評価には粒子フラックス予測に加え、衛星の帯電・放電リスクのモデル化も
必要であるため、それに向けた基礎知識の習得にも積極的に取り組みたい。
「NICTの研究者と協力してひまわり10号搭載用宇宙環境センサの開発」というテーマについて、
申請者はこれまで機器開発の経験を有していないものの、地上試験・評価実験・搭載前データ解析などに
積極的に携わり、宇宙機器開発の実務経験を段階的に積み重ねていきたいと考えている。



\vspace{1em}
\noindent
\bf{参考文献}








\end{document}
